\documentclass[12pt]{article}
\usepackage[margin=1in]{geometry}
\usepackage{amsfonts,amsmath,amssymb,amsthm,mathtools}
\newtheorem{definition}{Definition}[subsection]
\newtheorem{proposition}{Proposition}[subsection]
\newtheorem{theorem}{Theorem}[subsection]



\begin{document}

\section{Introduction}
\begin{definition}
	Let $X$ be an arbitrary set. A collection $\mathcal{A}$ of subsets of $X$ is called an \underline{algebra} if it satisfies the following four conditions:
	\begin{enumerate}
		\item $\emptyset \in \mathcal{A}$
		\item $A \in \mathcal{A} \Rightarrow A^\mathsf{c} \in \mathcal{A}$
		\item for each finite sequence, $A_1, A_2, \ldots, A_n \in \mathcal{A}$, we have $\bigcup_{i=1}^n A_i \in \mathcal{A}$
		\item for each finite sequence, $A_1, A_2, \ldots, A_n \in \mathcal{A}$, we have $\bigcap_{i=1}^n A_i \in \mathcal{A}$
	\end{enumerate}
\end{definition}




\begin{definition}
	Let $X$ be an arbitrary set. A collection $\mathcal{A}$ of subsets of $X$ is called a \underline{$\sigma$-algebra} if it satisfies the following four conditions:
	\begin{enumerate}
		\item $\emptyset \in \mathcal{A}$
		\item $A \in \mathcal{A} \Rightarrow A^\mathsf{c} \in \mathcal{A}$
		\item for each countable sequence, $A_1, A_2, \ldots \in \mathcal{A}$, we have $\bigcup_{i=1}^\infty A_i \in \mathcal{A}$
		\item for each countable sequence, $A_1, A_2, \ldots \in \mathcal{A}$, we have $\bigcap_{i=1}^\infty A_i \in \mathcal{A}$
	\end{enumerate}
\end{definition}


\begin{proposition}
	The intersection of an arbitrary non-empty collection of $\sigma$-algebras on $X$ is a $\sigma$-algebra on $X$.
\end{proposition}

\begin{definition}
	The \underline{Borel $\sigma$-algebra} on $\mathbb{R}^d$ is the $\sigma$-algebra generated by the collection of all open subsets of $\mathbb{R}^d$.
	Equivalently, it is generated by
	\begin{enumerate}
		\item all open subsets
		\item all closed subsets 
		\item all closed half spaces of the form $\{(x_1, \ldots, x_d) \in \mathbb{R}^d: x_i \leq b_i\}$, where $b_i \in \mathbb{R}$ for $1 \leq i \leq d$
		\item all closed rectangles of the form $\{(x_1, \ldots, x_d) \in \mathbb{R}^d: a_i \leq x_i \leq b_i\}$, where $a_i, b_i \in \mathbb{R}$ for $1 \leq i \leq d$
	\end{enumerate}
	of $\mathbb{R}^d$
\end{definition}

\begin{definition}
	A function from a $\sigma$-algebra $\mathcal{A}$ to $\mathbb{R}$ is \underline{countably additive} if for any countable sequence $A_1, A_2, \ldots \in \mathcal{A}$ of pairwise disjoint sets, we have
	\begin{equation*}
		f\left(\bigcup_{i=1}^\infty A_i\right) = \sum_{i=1}^\infty f(A_i)
	\end{equation*}
\end{definition}

\begin{definition}
	A \underline{measure} on a $\sigma$-algebra $\mathcal{A}$ is a function $\mu: \mathcal{A} \to [0, \infty]$ that is countably additive. We call the triple $(X, \mathcal{A}, \mu)$ a \underline{measure space}.
\end{definition}

\begin{definition}
	A measure space is finite if $\mu(X) < \infty$. A measure space is $\sigma$-finite if $X$ can be written as a countable union of sets of finite measure. That is, there exists a sequence $X_1, X_2, \ldots \in \mathcal{A}$ such that $X = \bigcup_{i=1}^\infty X_i$ and $\mu(X_i) < \infty$ for all $i$.
\end{definition}

\begin{proposition}
	$$\mu \left(\bigcup_{i=1}^\infty A_i\right) \leq \sum_{i=1}^\infty \mu(A_i)$$
\end{proposition}

\section{Outer Measure}
\begin{definition}
	Let $X$ be an arbitrary set. A function $\mu^*: \mathcal{P}(X) \to [0, \infty]$ is called an \underline{outer measure} if it satisfies the following three conditions:
	\begin{enumerate}
		\item $\mu^*(\emptyset) = 0$
		\item $A \subseteq B \Rightarrow \mu^*(A) \leq \mu^*(B)$
		\item for any countable sequence $A_1, A_2, \ldots \in \mathcal{P}(X)$, we have
		      \begin{equation*}
		      	\mu^*\left(\bigcup_{i=1}^\infty A_i\right) \leq \sum_{i=1}^\infty \mu^*(A_i)
		      \end{equation*}
	\end{enumerate}
\end{definition}

\begin{definition}
	A set $A \subseteq \mathbb{R}^d$ is \underline{$\mu^*$-measurable} if for any $E \subseteq \mathbb{R}^d$, we have
	\begin{equation*}
		\mu^*(E) = \mu^*(E \cap A) + \mu^*(E \cap A^\mathsf{c})
	\end{equation*}
\end{definition}

\begin{definition}
	For each subset $A$ of $\mathbb{R}^d$, let be the set of all infinite sequences of bounded, open intervals of the form $(a_1, b_1) \times \cdots \times (a_d, b_d)$ such that $A \subseteq \bigcup_{i=1}^\infty (a_1, b_1) \times \cdots \times (a_d, b_d)$. We
	define the \underline{Lebesgue outer measure} $\lambda^*:\mathcal{P}(\mathbb{R}^d) \to [0, \infty]$ by
	$$ A \mapsto \inf\left\{\sum_{i=1}^\infty \prod_{j=1}^d (b_{ij} - a_{ij}): (a_{ij}, b_{ij})_{i,j \in \mathbb{N}} \in \mathcal{I}_A\right\}$$
\end{definition}

\begin{proposition}
	The Lebesgue outer measure is an outer measure. It assigns to each d-dimensional rectangle the product of its side lengths.
\end{proposition}

\begin{proposition}
	Every Borel subset of $\mathbb{R}_d$ is Lebesgue measurable. We denote the collection of all Lebesgue measurable subsets of $\mathbb{R}_d$ by $\mathcal{M}_{\lambda^*}$.
\end{proposition}


\begin{proposition}
	Let $\mu^*$ be an outer measure on $X$. Then the collection of all $\mu^*$-measurable sets, denoted by $\mathcal{M}_{\mu^*}$, is a $\sigma$-algebra on $X$ and $\mu^*$ restricted to this $\sigma$-algebra is a measure.
	In particular, $\mathcal{M}_{\lambda^*}$ is a $\sigma$-algebra on $\mathbb{R}^d$ and $\lambda^*$ restricted to this $\sigma$-algebra is a measure. We call $\lambda^*$ restricted to $\mathcal{M}_{\lambda^*}$ the \underline{Lebesgue measure}, and denote it by $\lambda$.
\end{proposition}

\begin{definition}
	A finite or $\sigma$-finite measure $\mu$ on a $\sigma$-algebra $\mathcal{A}$ is \underline{continuous} if for all $x\in X$, we have $\mu(\{x\}) = 0$.
\end{definition}

\begin{definition}
	A finite or $\sigma$-finite measure $\mu$ on a $\sigma$-algebra $\mathcal{A}$ is \underline{discrete} if there is a countable subset $D$ of $X$ such that $\mu(D^\mathsf{c}) = 0$.
\end{definition}

\begin{definition}
	Let $\mu$ be a finite measure on a $(\mathbb{R}^d, \mathcal{B}(\mathbb{R}^d))$. Define a function $F_\mu: \mathbb{R}^d \to [0, \infty]$ by
	$$F_\mu(x) = \mu((-\infty, x])$$
	We call $F_\mu$ the \underline{distribution function} of $\mu$.
\end{definition}
\begin{proposition}
	The distribution function is bounded, non-decreasing, and right-continuous.
	$$ \lim_{x \to -\infty} F_\mu(x) = 0$$
\end{proposition}

\begin{theorem}
	Let $A$ be a Lebesgue-measurable subset of $\mathbb{R}^d$. Then
	$$\lambda(A) = \inf\left\{\lambda(U): A \subseteq U, U \text{ is open}\right\}= \sup\left\{\lambda(K): K \subseteq A, K \text{ is compact}\right\}$$
\end{theorem}

\begin{theorem}
	Any Borel subset of $\mathbb{R}^d$ is the union of a sequence of disjoint bounded Borel sets.
\end{theorem}

\begin{definition}
	The \underline{Cantor set} is the set of all $x \in [0, 1]$ that can be written as a ternary expansion using only the digits 0 and 2.
\end{definition}
\begin{proposition}
	The Cantor set is compact, has the cardinality of the continuum, and has Lebesgue measure 0.
\end{proposition}
\begin{definition}
	$\operatorname{diff}(A)\vcentcolon= \{x - y: x, y \in A\}$.
\end{definition}
\begin{proposition}
	If $A$ is Lebesgue measurable and $\lambda (A)>0$, then $\operatorname{diff}(A)$ includes an open interval that contains 0.
\end{proposition}

\section{Completeness}
\begin{definition}
	A measure space $(X, \mathcal{A}, \mu)$ is \underline{complete} if for all $A \in \mathcal{A}$ with $\mu(A) = 0$, we have $B \in \mathcal{A}$ for all $B \subseteq A$.
\end{definition}
\begin{definition}
	The subset $B$ is \underline{$\mu$ -negligible} if there exists $A\subseteq X$ with $A \in \mathcal{A}$ and $B \subseteq A$ such that $\mu(A) = 0$.
\end{definition}
\begin{proposition}
	if $\mu^*$ is is an outer measure on the set $X$ and if $\mathcal{M}_{\mu^*}$ is the $\sigma$-algebra of all $\mu^*$-measurable subsets of $X$, then the restriction of $\mu^*$ to $\mathcal{M}_{\mu^*}$ is complete.
\end{proposition}

\end{document}
